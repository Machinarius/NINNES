\documentclass[10pt]{beamer}

\usetheme[progressbar=frametitle]{metropolis}
\usepackage{appendixnumberbeamer}

\usepackage{tikz}
\usepackage{booktabs}
\usepackage[scale=2]{ccicons}

\usepackage{pgfplots}
\usepgfplotslibrary{dateplot}

\usepackage{xspace}
\newcommand{\themename}{\textbf{\textsc{metropolis}}\xspace}

\usepackage{caption}
\usepackage{subcaption}
\captionsetup{labelformat=empty}

\title{NINNES}
\subtitle{dotNet and Roslyn behind the Curtains}
% \date{\today}
\date{November 8, 2018}
\author{Germán Valencia - @XMachinarius}
\institute{Growth Acceleration Partners}
\titlegraphic{\hfill\includegraphics[height=1.5cm]{img/gap_logo.png}}
\logo{\includegraphics[height=1.5cm]{img/gap_logo.png}}
\begin{document}

\maketitle

\begin{frame}{Table of contents}
  \setbeamertemplate{section in toc}[sections numbered]
  \tableofcontents[hideallsubsections]
\end{frame}

\section{Introduction}
\begin{frame}{What are we here for?}
    \begin{itemize}
        \item NINNES project
        \item dotNet compilation and execution internals
        \item Roslyn internals
        \item Cool demo
    \end{itemize}
\end{frame}

\begin{frame}{NINNES Project}
    \begin{enumerate}
        \item Bringing C\# into NES game programming
        \item Customizing the dotNet compilation process
        \item Programming CPU-equivalent libraries
    \end{enumerate}
\end{frame}

\begin{frame}{C\# Code Tailoring}
    \begin{figure}
        \centering
        \includegraphics[height=190pt]{img/scissors.jpg}
        \caption{Cutting C\# down to size \\ \tiny{\href{Photo by Pixabay, taken from Pexels}{https://www.pexels.com/photo/fabric-scissors-needle-needles-scissors-461035/}}}
    \end{figure}
\end{frame}

\begin{frame}{C\# Code Tailoring}
    \begin{enumerate}
        \item Remove invalid instructions (and substitute them where possible)
        \item Assist the programmer with keeping his C\# code compatible with the NES CPU
        \item Invalidate code that is otherwise perfectly fine
    \end{enumerate}
\end{frame}

\begin{frame}{NES dotNet Shims Platform}
    \begin{figure}
        \centering
        \includegraphics[width=\textwidth]{img/bridge.jpg}
        \caption{Bridging the gap to the unknown \\ \tiny{\href{Photo by Nikolai Ulltang from Pexels}{https://www.pexels.com/photo/architecture-bridge-fog-ocean-285283/}}}
    \end{figure}
\end{frame}

\begin{frame}{NES dotNet Shims Platform}
    \begin{enumerate}
        \item NuGet Package exposing dotNet-friendly abstractions over the NES CPU, APU and PPU
        \item F5 into a CLR-based rendering of the code, XNA-style
    \end{enumerate}
\end{frame}

\begin{frame}{C\# Compilation to the NES CPU}
    \begin{figure}
        \centering
        \includegraphics[width=160pt]{img/cardmaster.jpg}
        \caption{One Ace up the sleeve. \\ \tiny{\href{Photo by Nikolai Ivanov from Pexels}{https://www.pexels.com/photo/actor-adult-business-cards-547593/}}}
    \end{figure}
\end{frame}

\begin{frame}{Why?}
The main objective of this project is to showcase Roslyn's flexibility for custom scenarios with an extreme case study relative to the expected use cases on the usual day to day development, so as to produce a library of Roslyn customization examples covering a wide range of scenarios. 

And for 1337 internet points, of course.
\end{frame}

\section{From Visual Studio to F5, dotNet behind the curtains}
\begin{frame}{The show begins}
    \begin{figure}
        \centering
        \includegraphics[width=250pt]{img/curtains.jpg}
        \caption{From idea to execution, a play by Microsoft \\ \tiny{\href{Photo by Monica Silvestre from Pexels}{https://www.pexels.com/photo/people-at-theater-713149/}}}
    \end{figure}
\end{frame}

\begin{frame}{C\#, the Idiom and Culture}
    The programming language we all know and love.
\end{frame}

\begin{frame}{BCL, the Acting Academy}
    \begin{itemize} 
        \item Base Class Library
        \item Basically, the System namespace and basic types
        \item Extended with Operating System service wrappers like Windows Forms and WPF via P/Invoke
    \end{itemize}
\end{frame}

\begin{frame}{Roslyn, Costumer Extraordinaire}
    \begin{itemize}
        \item ".Net Compiler Platform", excepts everybody calls it Roslyn per the project Codename
        \item Developed as a Tour de Force to show the maturity of C\# and dotNet
        \item "Strengthening the ecosystem and becoming the best tooled language on the planet" - Mads Torgensen, C\# designer
    \end{itemize}
\end{frame}

\begin{frame}{Roslyn, Costumer Extraordinaire}
    \begin{itemize}
        \item Not only a compiler, a dotNet Compilation SDK
        \item Exposes multiple services to external code, allowing for extension and modification
        \item Used by Visual Studio for Windows, Mac and VS Code/Sublime Text/Atom... (Via OmniSharp)
    \end{itemize}
\end{frame}

\begin{frame}{MSIL, the Script}
    \begin{itemize}
        \item Provides dotNet with ABI/Platform independence
        \item Machine-agnostic, Portable language
        \item JIT-ed into machine code
    \end{itemize}
\end{frame}

\begin{frame}{MSBuild, the Playwright}
    \begin{itemize}
        \item Source of Truth for the dotNet project models
        \item Coordinator of project-associated tooling
    \end{itemize}
\end{frame}

\begin{frame}{RyuJIT, the Director}
    RyuJIT falls in the Just In Time model of dotNet execution as the translator from MSIL to native CPU code.
\end{frame}

\begin{frame}{CLI, the Stage}
    \begin{itemize}
        \item ECMA 335, standard since dotNet 1.0
        \item Open Standard describing the execution of dotNet/MSIL code
        \item Implemented by Microsoft with the CLR and by Xamarin with Mono
    \end{itemize}
\end{frame}

\begin{frame}{DLR, the Improv Guide}
    \begin{itemize}
        \item Dynamic Language Runtime
        \item Extendes the CLI implementations to add support for dynamic languages like IronPython and IronRuby
    \end{itemize}
\end{frame}

\begin{frame}{dotNet Native, the Cameraman}
    Invokes RyuJIT ahead of time to produce a native yet CLI-dependant image for startup-time sensitive workloads
\end{frame}

\section{From Visual Studio to the NES}
\begin{frame}{The Limits of the NES}
The NES runs on a MOS6502 8-bit CPU implementation from Ricoh, albeit without the decimal/floating point mode enabled. This brings forth several limitations:
\begin{itemize}
    \item No floating point calculations
    \item No multiplication or division implemented in hardware
    \item No Operating System
    \item Limited memory access capabilities
\end{itemize}
\end{frame}

\begin{frame}{Removing dotNet from C\#, 8 bits at a time}
To be able to fit a C\# program inside the NES some concessions must be made, including:
\begin{itemize}
	\item No Garbage Collection
	\item No Threading
	\item No Operating System services
	\item No DLR
	\item No Reflection
	\item No Sockets/Communication
\end{itemize}
\end{frame}

\begin{frame}{MSIL and 8-bit CPUs don't mix}
Bringing C\# compilation output to a MOS6502-compatible format may entail
\begin{itemize}
	\item Direct ASM transpilation from Roslyn C\# ASTs
	\item C++ transpilation from Roslyn C\# ASTs for compilation with a MOS6502-compatible compiler
	\item MSIL recompilation with LLILC
	\item Researching the work of the .Net Micro Framework
\end{itemize}
\end{frame}

\section{Roslyn is a Friend, not an Enemy}
\begin{frame}{General Roslyn Services}
    \begin{figure}
        \center
        \includegraphics[width=250pt]{img/RoslynServices.png}
        \caption{Some of the services exposed by Roslyn \\ \tiny{\href{Taken from DotNetCurry}{https://www.dotnetcurry.com/csharp/1258/dotnet-platform-compiler-roslyn-overview}}}
    \end{figure}
\end{frame}

\begin{frame}{Compiler basics - What is code?}
English is the language of the human, Assembler OpCodes is the language of the machine. Compilers must act as a bridge in between these two realms of knowledge, and they too have their own abstraction: The Abstract Syntax Tree.
\end{frame}

\begin{frame}[fragile]{Compiler basics - Code Sample}
\begin{verbatim}
var x = 2;
var y = 3;
x *= y;
\end{verbatim}
\end{frame}

\begin{frame}{Compiler basics - Abstract Syntax Tree}
    \begin{figure}
        \center
        \includegraphics[width=\textwidth]{img/RoslynASTSample.png}
        \caption{A basic Roslyn AST \\ \tiny{Own work}}
    \end{figure}
\end{frame}

\begin{frame}{Roslyn Code Model}
Roslyn Design Goals:

\begin{block}{Speed}
    Analysis must be performed incrementally and in real time
\end{block}

\begin{block}{Memory Economy}
    Visual Studio can already be daunting for some machines
\end{block}
\end{frame}

\begin{frame}{Roslyn Basics - Code Diagnostics}
Two words: CUSTOM SQUIGGLIES
\end{frame}

\section{DEMO}

\begin{frame}{Thank you!}
\begin{block}{Germán Valencia}
    @XMachinarius \\ Growth Acceleration Partners
\end{block}
\end{frame}

{ % Source: https://tex.stackexchange.com/questions/3915/image-on-full-slide-in-beamer-package
    \setbeamertemplate{navigation symbols}{}
    \begin{frame}[plain]
        \begin{tikzpicture}[remember picture,overlay]
            \node[at=(current page.center)] {
                \includegraphics[width=\paperwidth]{img/NetConfOutroBackground.jpg}
            };
        \end{tikzpicture}
     \end{frame}
}
\end{document}
